\nonstopmode{}
\documentclass[letterpaper]{book}
\usepackage[times,inconsolata,hyper]{Rd}
\usepackage{makeidx}
\makeatletter\@ifl@t@r\fmtversion{2018/04/01}{}{\usepackage[utf8]{inputenc}}\makeatother
% \usepackage{graphicx} % @USE GRAPHICX@
\makeindex{}
\begin{document}
\chapter*{}
\begin{center}
{\textbf{\huge Package `UtilsCompassSV'}}
\par\bigskip{\large \today}
\end{center}
\ifthenelse{\boolean{Rd@use@hyper}}{\hypersetup{pdftitle = {UtilsCompassSV: Utility functions to plot and work with COMPASS output}}}{}
\begin{description}
\raggedright{}
\item[Type]\AsIs{Package}
\item[Title]\AsIs{Utility functions to plot and work with COMPASS output}
\item[Version]\AsIs{0.1.6}
\item[Author]\AsIs{Miguel Julio Rodo}
\item[Maintainer]\AsIs{Miguel Julio Rodo }\email{rdxmig002@myuct.ac.za}\AsIs{}
\item[Description]\AsIs{Primary function thus far is plot_compass, 
which plots posterior probabilities for each cytokine combination
    using more concise alternatives to the COMPASS::plot function's grid. 
    Also has `convert_cyt_combn_format`, for converting between ``standard'' and COMPASS
    cytokine combination formats. }
\item[License]\AsIs{MIT + file LICENSE}
\item[Encoding]\AsIs{UTF-8}
\item[LazyData]\AsIs{true}
\item[Depends]\AsIs{R (>= 2.10)}
\item[Suggests]\AsIs{testthat (>= 3.0.0)}
\item[Config/testthat/edition]\AsIs{3}
\item[Imports]\AsIs{purrr, magrittr, stringr, tibble, rlang, tidyr, cowplot,
COMPASS, RColorBrewer, ggplot2}
\item[RoxygenNote]\AsIs{7.1.2}
\item[NeedsCompilation]\AsIs{no}
\end{description}
\Rdcontents{Contents}
\HeaderA{.get\_col\_vec\_grp}{Get colours for groups for boxplots}{.get.Rul.col.Rul.vec.Rul.grp}
%
\begin{Description}
Get colours for groups for boxplots
\end{Description}
%
\begin{Usage}
\begin{verbatim}
.get_col_vec_grp(plot_prob_fill = NULL, .grp = NULL)
\end{verbatim}
\end{Usage}
%
\begin{Arguments}
\begin{ldescription}
\item[\code{plot\_prob\_fill}] character. If a colour, then the boxplots are filled according
to the specified value. If a named vector of colours names correspond to names
of the bottom-most level of the \code{c\_obj} lists, then boxplots are filled
accordingly.  Otherwise, boxplots are filled according to names of lower-most list but using

\item[\code{.grp}] character vector. Names of groups in COMPASS object.
\end{ldescription}
\end{Arguments}
%
\begin{Value}
A named character vector, where names are names of groups
and elements are colours for the groups
\end{Value}
%
\begin{Examples}
\begin{ExampleCode}
.get_col_vec_grp(plot_prob_fill = "red", .grp = c("grp1", "grp2"))
.get_col_vec_grp(plot_prob_fill = NULL, .grp = c("grp1", "grp2"))
.get_col_vec_grp(plot_prob_fill = c("grp1" = "red", "grp2" = "orange"))
\end{ExampleCode}
\end{Examples}
\HeaderA{.get\_col\_vec\_grp\_null}{Get colour vector for groups if plot\_prob\_fill is NULL}{.get.Rul.col.Rul.vec.Rul.grp.Rul.null}
%
\begin{Description}
Get colour vector for groups if plot\_prob\_fill is NULL
\end{Description}
%
\begin{Usage}
\begin{verbatim}
.get_col_vec_grp_null(.grp)
\end{verbatim}
\end{Usage}
\HeaderA{.plot\_compass\_pp}{Plot COMPASS posterior probabilites}{.plot.Rul.compass.Rul.pp}
%
\begin{Description}
Plot COMPASS posterior probabilites
\end{Description}
%
\begin{Usage}
\begin{verbatim}
.plot_compass_pp(
  c_obj,
  dir_save,
  prob_min,
  quant_min,
  silent,
  cyt_order,
  plot_prob_fill,
  facet,
  cyt_lab,
  boxplot_width,
  font_size
)
\end{verbatim}
\end{Usage}
%
\begin{Arguments}
\begin{ldescription}
\item[\code{c\_obj}] a named list of COMPASSResult objects. Provides data to plot.

\item[\code{dir\_save}] character. Where to save the output. Default is working directory.

\item[\code{prob\_min}] [0,1]. Specify the minimum probability of a response for the minimum quantile
of samples that a cytokine combination must have to be included. For example,
if \code{prob\_min == 0.5} and quant\_min == 0.1, then only cytokine combinations for which
at least 10
Default is \code{0.8} and \code{0.25}.

\item[\code{quant\_min}] [0,1]. Specify the minimum probability of a response for the minimum quantile
of samples that a cytokine combination must have to be included. For example,
if \code{prob\_min == 0.5} and quant\_min == 0.1, then only cytokine combinations for which
at least 10
Default is \code{0.8} and \code{0.25}.

\item[\code{silent}] logical. If \code{TRUE}, then any warnings that would have been otherwise given
are not. Default is \code{FALSE}.

\item[\code{cyt\_order}] character. If not \code{NULL}, then the order in which cytokines are arranged
vertically along the annotation grid is given by this vector (with first element going at the bottom).
If \code{NULL}, then cytokines are ordered by their order in COMPASS output. Default is \code{NULL}.

\item[\code{plot\_prob\_fill}] character. If a colour, then the boxplots are filled according
to the specified value. If a named vector of colours names correspond to names
of the bottom-most level of the \code{c\_obj} lists, then boxplots are filled
accordingly.  Otherwise, boxplots are filled according to names of lower-most list but using

\item[\code{facet}] logical vector. Whether to facet or save individual plots for each
group in \code{c\_obj}. If \code{TRUE} only, then only a faceted version is saved.
If \code{FALSE} only, then plots are saved individually. If \code{c(TRUE, FALSE)},
then both the facted and individual plots are saved.

\item[\code{cyt\_lab}] function. A function that takes the cytokine names as input and returns the
value desired to be plotted along the y-axis of the grid plot. If not \code{NULL}, then
it is supplied when creating the grid plot to the \code{scale\_y\_continuous} function
via the \code{labels} parameter. For example, if we have two cytokines, \code{"IFNg"} and
\code{"TNF"}, but we want to display \code{"IFNg"} with the Greek gamma symbol, then we can
set \code{cyt\_lab} equal to the following: \code{cyt\_lab = function(cyt) purrr::map(cyt, function(cyt\_ind)\{switch(cyt\_ind, "IFNg" = bquote(paste(plain(paste("IFN")), gamma)), cyt\_ind)\})}.
This will change the label for \code{"IFNg"} but leave all the others as is.
\end{ldescription}
\end{Arguments}
%
\begin{Value}
A named list with names 'p\_probs' and 'p\_grid',
corresponding to the list of ggplot2 plots of posterior plots
and the heat map of the corresponding cytokine combination
labels, respectively.
\end{Value}
\HeaderA{.plot\_compass\_scores}{Plot COMPASS PFS and FS scores}{.plot.Rul.compass.Rul.scores}
%
\begin{Description}
Plot COMPASS PFS and FS scores
\end{Description}
%
\begin{Usage}
\begin{verbatim}
.plot_compass_scores(
  c_obj,
  plot_prob_fill,
  boxplot_width,
  plot_scores_lims_y,
  font_size
)
\end{verbatim}
\end{Usage}
%
\begin{Arguments}
\begin{ldescription}
\item[\code{c\_obj}] a named list of COMPASSResult objects. Provides data to plot.

\item[\code{plot\_prob\_fill}] character. If a colour, then the boxplots are filled according
to the specified value. If a named vector of colours names correspond to names
of the bottom-most level of the \code{c\_obj} lists, then boxplots are filled
accordingly.  Otherwise, boxplots are filled according to names of lower-most list but using

\item[\code{plot\_scores\_lims\_y}] numeric vector of length 2.
If not \code{NULL}, specifies the y-axis range for the scores plots.
Default is \code{NULL}.
\end{ldescription}
\end{Arguments}
%
\begin{Value}
A named list with names set to the names
of \code{c\_obj}, where each element is the corresponding
boxplot of PFS and FS scores.
\end{Value}
\HeaderA{.save\_layout}{Create a layout based on post probs, cyt heat map and scores}{.save.Rul.layout}
%
\begin{Description}
Create a layout based on post probs, cyt heat map and scores
\end{Description}
%
\begin{Usage}
\begin{verbatim}
.save_layout(
  ind,
  p_list_pp,
  p_list_scores = NULL,
  prop_pp,
  shift_plot_heatmap_x,
  shift_plot_scores_y,
  shift_plot_pp_y,
  height,
  width,
  dir_save,
  file,
  save_format,
  label,
  n_col,
  shift_label,
  font_size_labels
)
\end{verbatim}
\end{Usage}
%
\begin{Arguments}
\begin{ldescription}
\item[\code{ind}] logical.
If \code{TRUE}, then sets of posterior probability
and PFS plots are saved individually.

\item[\code{p\_list\_pp}] list of \code{ggplot2} objects.
Posterior probability plots and associated
cytokine-label grids, as returned by \code{UtilsCompassSV:::.plot\_compass\_pp}.

\item[\code{p\_list\_scores}] list of \code{ggplot2} objects.
PFS and FS response plots, as
returned by \code{UtilsCompassSV:::.plot\_compass\_scores}.

\item[\code{prop\_pp}] numeric vector, values in \code{[0,1]}. Specifies coordinate-wise the proportion of plot region
for a single group in \code{c\_obj} (if \code{c\_obj} is a list; otherwise it's simply the proportion of the
plot) that is devoted (before applying \code{shift\_plot\_pp\_y}) to the plot of posterior probabilities.
The first element is the proportion allocated to the probability plot along the x-axis, and (\code{1-prop\_pp[1]}) is
then the space allocated to the scores plot. The first element is set equal to 0 if there is no scores plot.
The second element is the proportion allocate to the probabilty plot along the y-axis, and (\code{1-prop\_pp[1]}) is
then the space allocated to the cytokine indicator grid.

\item[\code{shift\_plot\_heatmap\_x}] [0,1]. Extent to shift prob\_plot. Increasing it from 0 moves the start of
the probability plot further to the right. Maximum value is 1 (at which point the plot will
effectively be pushed off the plotting surface). Useful to increase to a value such as 0.01
if the cytokine names are long and push the labelling grid too far to the right. Tweak as required.
Default is \code{0}.

\item[\code{shift\_plot\_scores\_y}] numeric vector of length two, restricted to [-1,1]. Specifies the amount to
squeeze the scores plot in. The first element controls the bottom position of the plot,
and the second the top. For both elements, a positive value means moving upwards. A value of zero corresponds no shift.
Typically the bottom element and top element should both be shifted down (and therefore have negative values).
Only applies if type includes both \code{"pp"} and \code{"scores"}.
Default is \code{c(0,0)}.

\item[\code{shift\_plot\_pp\_y}] [0,1]. Specifies amount by which to shift the upper point of the probability plot upwards.
Positive values make it larger. Typically you want to shift this down, if anything, so use negative values.
Default is 0.

\item[\code{height}, \code{width}] numeric. Height and width of saved plot(s).

\item[\code{dir\_save}] character. Where to save the output. Default is working directory.

\item[\code{save\_format}] "pdf" or "png". Plot device to use. Default is "png".

\item[\code{label}] logical. If \code{TRUE} and \code{c\_obj} is a list, then the names of
elements in \code{c\_obj} are used as labels for sub-figures.
If \code{c\_obj} is not a list, then no label is printed, regardless of the
value of \code{label}. Default is \code{TRUE}.

\item[\code{shift\_label}] numeric vector, values in \code{[0,1]}. Specifies amount by which to shift
the labels of the plot (if used). Default is \code{c(0.05, -0.04)}.
\end{ldescription}
\end{Arguments}
\HeaderA{.save\_plot}{Save an individual layout}{.save.Rul.plot}
%
\begin{Description}
Save an indivdual plot layout.
\end{Description}
%
\begin{Usage}
\begin{verbatim}
.save_plot(p, height, width, dir_save, file, save_format, n_row, n_col)
\end{verbatim}
\end{Usage}
%
\begin{Arguments}
\begin{ldescription}
\item[\code{p}] object of class \code{gg}. Plot to save.

\item[\code{dir\_save}] character. Where to save the output. Default is working directory.

\item[\code{save\_format}] "pdf" or "png". Plot device to use. Default is "png".

\item[\code{n\_row}, \code{n\_col}] integer. Number of objects plotted per row and column.
Determines \code{width} and \code{height} if they are \code{NULL}.
\end{ldescription}
\end{Arguments}
\HeaderA{convert\_cyt\_combn\_format}{Convert cytokine combination format to or from COMPASS format}{convert.Rul.cyt.Rul.combn.Rul.format}
%
\begin{Description}
Convert cytokine combination format between COMPASS format and "standard " +/- format.
See \code{cyt\_combn} parameter for details. NOTE: At present only converts from COMPASS to standard format.
\end{Description}
%
\begin{Usage}
\begin{verbatim}
convert_cyt_combn_format(
  cyt_combn,
  to,
  force = FALSE,
  silent = FALSE,
  check = TRUE,
  lab = NULL
)
\end{verbatim}
\end{Usage}
%
\begin{Arguments}
\begin{ldescription}
\item[\code{cyt\_combn}] character vector. Cytokine combination, specified either in
"<cyt><+-><cyt><+-><cyt><+->..." (e.g. IFNg+IL2-TNF+) or
COMPASS "<!><cyt>\&<!><cyt>\&<!><cyt>..." (e.g. IFNg\&!IL2\&TNF) format.

\item[\code{to}] 'compass' or 'std'. Format to convert to. If \code{'compass'},
then output format is COMPASS format (specified above). If \code{'std'},
then output format is standard format (specified above).

\item[\code{force}] logical. If \code{TRUE}, then the code to convert to the specified format will be run.
If \code{FALSE}, then checks are made to ensure that conversion is not done if already in the
requested format. Default is \code{FALSE}.

\item[\code{silent}] logical. If \code{FALSE}, then warnings are printed if \code{force == FALSE} and
it is detected that \code{cyt\_combn} already appears to be in the requested format. Default is \code{FALSE}.

\item[\code{check}] logical. If \code{TRUE}, then the converted format is checked that each element has
the same number of cytokines. Note that check is only performed if the conversion is attempted. Default is \code{TRUE}.

\item[\code{lab}] named character vector. Names are names for markers/channels as found in the cytokine combination,
and elements are corresponding names/channels. If \code{NULL}, then no labelling is done. Default is \code{NULL}.
\end{ldescription}
\end{Arguments}
%
\begin{Value}
character vector.
\end{Value}
%
\begin{Examples}
\begin{ExampleCode}
convert_cyt_combn_format(c("IFNg&!IL2"), to = "std")
convert_cyt_combn_format(c("IFNg+IL2-"), to = "compass")
\end{ExampleCode}
\end{Examples}
\HeaderA{plot\_compass}{Plot COMPASS output}{plot.Rul.compass}
%
\begin{Description}
Plot COMPASS posterior probabilities as boxplots, and the PFS scores if desired.
A grid of figures is created whenever multiple COMPASS objects are plotted simultaneous, with
y-axis scales and cytokine combinations displayed fixed across plots.

' @param c\_obj object of class "COMPASSResult", or a list of such objects. Provides
COMPASS data to plot.
\end{Description}
%
\begin{Usage}
\begin{verbatim}
plot_compass(
  c_obj,
  dir_save = getwd(),
  type = c("pp", "scores"),
  save = TRUE,
  save_format = "png",
  prob_min = 0.8,
  quant_min = 0.25,
  boxplot_width_scores = NULL,
  boxplot_width_pp = NULL,
  silent = FALSE,
  cyt_order = NULL,
  file_grid = NULL,
  file_ind = NULL,
  plot_prob_fill = NULL,
  shift_plot_heatmap_x = 0,
  shift_plot_scores_y = c(0, 0),
  shift_plot_pp_y = 0,
  shift_label = c(0.05, -0.04),
  prop_pp = c(0.7, 0.7),
  label = TRUE,
  return_plot_list = TRUE,
  facet = FALSE,
  n_col = NULL,
  cyt_lab = NULL,
  save_grid = TRUE,
  height_grid = NULL,
  width_grid = NULL,
  save_ind = FALSE,
  font_size = 14,
  height_ind = NULL,
  width_ind = NULL,
  plot_scores_lims_y = NULL,
  font_size_labels = 14
)
\end{verbatim}
\end{Usage}
%
\begin{Arguments}
\begin{ldescription}
\item[\code{dir\_save}] character. Where to save the output. Default is working directory.

\item[\code{type}] "pp" and/or "scores". Specifies response type(s) to plot.
If "pp" is included, then posterior probabilities of individual cytokine combinations are plotted.
If "scores" is included, then the PFS and FS responses are included. Note that, at this stage,
"pp" needs to be included, and so is added even if missing from \code{type}. Default is \code{c("pp", "scores")}.

\item[\code{save}] logical. If \code{TRUE}, then plots are saved. Default is \code{TRUE}.

\item[\code{save\_format}] "pdf" or "png". Plot device to use. Default is "png".

\item[\code{prob\_min}, \code{quant\_min}] [0,1]. Specify the minimum probability of a response for the minimum quantile
of samples that a cytokine combination must have to be included. For example,
if \code{prob\_min == 0.5} and quant\_min == 0.1, then only cytokine combinations for which
at least 10
Default is \code{0.8} and \code{0.25}.

\item[\code{boxplot\_width\_pp}, \code{boxplot\_width\_scores}] numeric. If not \code{NULL}, then supplied to \code{width} parameter of
\code{geom\_boxplot} for the posterior probability and scores plots, respectively. Purpose is
to force the widths of the boxplots to be constant across elements in \code{c\_obj}.
Default is \code{NULL}.

\item[\code{silent}] logical. If \code{TRUE}, then any warnings that would have been otherwise given
are not. Default is \code{FALSE}.

\item[\code{cyt\_order}] character. If not \code{NULL}, then the order in which cytokines are arranged
vertically along the annotation grid is given by this vector (with first element going at the bottom).
If \code{NULL}, then cytokines are ordered by their order in COMPASS output. Default is \code{NULL}.

\item[\code{file\_grid}, \code{file\_ind}] character (vector).
Names for grid plot and individual plots, respectively, to be saved as.
If \bsl{}codeis.null(file), then the grid plot is
simply named \code{compass\_boxplots\_grid}.
If \bsl{}codeis.null(file\_ind), then each element
has its name taken from the corresponding list name in \code{c\_obj}.
If \code{file\_ind} is named, then the names of \code{c\_obj}
are used to map onto elements of \code{file\_ind}.

\item[\code{plot\_prob\_fill}] character. If a colour, then the boxplots are filled according
to the specified value. If a named vector of colours names correspond to names
of the bottom-most level of the \code{c\_obj} lists, then boxplots are filled
accordingly.  Otherwise, boxplots are filled according to names of lower-most list but using

\item[\code{shift\_plot\_heatmap\_x}] [0,1]. Extent to shift prob\_plot. Increasing it from 0 moves the start of
the probability plot further to the right. Maximum value is 1 (at which point the plot will
effectively be pushed off the plotting surface). Useful to increase to a value such as 0.01
if the cytokine names are long and push the labelling grid too far to the right. Tweak as required.
Default is \code{0}.

\item[\code{shift\_plot\_scores\_y}] numeric vector of length two, restricted to [-1,1]. Specifies the amount to
squeeze the scores plot in. The first element controls the bottom position of the plot,
and the second the top. For both elements, a positive value means moving upwards. A value of zero corresponds no shift.
Typically the bottom element and top element should both be shifted down (and therefore have negative values).
Only applies if type includes both \code{"pp"} and \code{"scores"}.
Default is \code{c(0,0)}.

\item[\code{shift\_plot\_pp\_y}] [0,1]. Specifies amount by which to shift the upper point of the probability plot upwards.
Positive values make it larger. Typically you want to shift this down, if anything, so use negative values.
Default is 0.

\item[\code{shift\_label}] numeric vector, values in \code{[0,1]}. Specifies amount by which to shift
the labels of the plot (if used). Default is \code{c(0.05, -0.04)}.

\item[\code{prop\_pp}] numeric vector, values in \code{[0,1]}. Specifies coordinate-wise the proportion of plot region
for a single group in \code{c\_obj} (if \code{c\_obj} is a list; otherwise it's simply the proportion of the
plot) that is devoted (before applying \code{shift\_plot\_pp\_y}) to the plot of posterior probabilities.
The first element is the proportion allocated to the probability plot along the x-axis, and (\code{1-prop\_pp[1]}) is
then the space allocated to the scores plot. The first element is set equal to 0 if there is no scores plot.
The second element is the proportion allocate to the probabilty plot along the y-axis, and (\code{1-prop\_pp[1]}) is
then the space allocated to the cytokine indicator grid.

\item[\code{label}] logical. If \code{TRUE} and \code{c\_obj} is a list, then the names of
elements in \code{c\_obj} are used as labels for sub-figures.
If \code{c\_obj} is not a list, then no label is printed, regardless of the
value of \code{label}. Default is \code{TRUE}.

\item[\code{return\_plot\_list}] logical. If \code{TRUE}, then a named list of the plots used to create the figure are returned. The
first element is \code{"p\_grid"} for the cytokine grid plot. The second element is \code{"p\_probs"}  for
a list of the posterior probability plots. The third element is named \code{"p\_scores"}, and has sub-elements that
are the plots of the PFS and FS responses for each group. The \code{"p\_scores"} element is only supplied if \code{"scores"} is
in \code{"type"}.

\item[\code{facet}] logical vector. Whether to facet or save individual plots for each
group in \code{c\_obj}. If \code{TRUE} only, then only a faceted version is saved.
If \code{FALSE} only, then plots are saved individually. If \code{c(TRUE, FALSE)},
then both the facted and individual plots are saved.

\item[\code{cyt\_lab}] function. A function that takes the cytokine names as input and returns the
value desired to be plotted along the y-axis of the grid plot. If not \code{NULL}, then
it is supplied when creating the grid plot to the \code{scale\_y\_continuous} function
via the \code{labels} parameter. For example, if we have two cytokines, \code{"IFNg"} and
\code{"TNF"}, but we want to display \code{"IFNg"} with the Greek gamma symbol, then we can
set \code{cyt\_lab} equal to the following: \code{cyt\_lab = function(cyt) purrr::map(cyt, function(cyt\_ind)\{switch(cyt\_ind, "IFNg" = bquote(paste(plain(paste("IFN")), gamma)), cyt\_ind)\})}.
This will change the label for \code{"IFNg"} but leave all the others as is.

\item[\code{save\_grid}] logical. If \code{TRUE}, then a grid of all individual elements in \code{c\_obj} are saved.
Default is \code{TRUE}.

\item[\code{height\_grid}, \code{width\_grid}] numeric. Height and width, respectively, in centimetres of the saved figure (if saved). If
\code{NULL}, then appropriate values are guessed at and used. Default is \code{NULL}.

\item[\code{save\_ind}] logical. If \code{TRUE}, then plots for individual elements in \code{c\_obj} are saved.
Default is \code{FALSE}.

\item[\code{height\_ind}, \code{width\_ind}] numeric. Height and width, respectively, in centimetres of the saved figure (if saved),
when saving individual elements in \code{c\_obj}. If
\code{NULL}, then appropriate values are guessed at and used. Default is \code{NULL}.

\item[\code{plot\_scores\_lims\_y}] numeric vector of length 2.
If not \code{NULL}, specifies the y-axis range for the scores plots.
Default is \code{NULL}.
\end{ldescription}
\end{Arguments}
%
\begin{Value}
A list, where each element is a \code{ggplot2} object.
\end{Value}
%
\begin{Examples}
\begin{ExampleCode}
library(UtilsCompassSV)
data("c_obj_list", package = "UtilsCompassSV")
plot_compass(
  c_obj = c_obj_list,
  type = c("pp", "scores"),
  return_plot_list = FALSE,
  shift_plot_scores = c(-0.05, 0.05),
  shift_plot_pp_y = -0.075,
  shift_plot_heatmap_x = 0.052
)
# The plot will then be saved to the working directory.

# Can also use Greek symbols for cytokines:
get_cyt_lab <- function(cyt) {
  lapply(cyt, function(cyt_ind) {
    switch(cyt_ind,
      "IFNg" = bquote("IFN" ~ gamma),
      cyt_ind
    )
  })
}
plot_compass(c_obj_list[1],
  type = c("pp"),
  return = FALSE, shift_plot_scores = c(-0.05, 0.05), facet = FALSE,
  shift_plot_pp_y = -0.05, shift_plot_heatmap_x = 0.052,
  cyt_lab = get_cyt_lab
)
\end{ExampleCode}
\end{Examples}
\HeaderA{response\_prob}{Calculate COMPASS-derived overall responder probability}{response.Rul.prob}
%
\begin{Description}
Calculate the probability of responding to at least one
cytokine combination, assuming independence of the probability estimates
between cytokine combinations.
\end{Description}
%
\begin{Usage}
\begin{verbatim}
response_prob(c_obj, exc = NULL)
\end{verbatim}
\end{Usage}
%
\begin{Arguments}
\begin{ldescription}
\item[\code{c\_obj}] object of class 'COMPASSResult'. The posterior probabilities
for individual cytokine combinations are obtained here.

\item[\code{exc}] character vector. Specifies cytokine combination(s) to exclude.
If \code{NULL}, then none except the all-negative population are excluded.
Default is \code{NULL}.
\end{ldescription}
\end{Arguments}
%
\begin{Details}
Calculates the probability of an individual responding to
at least one cytokine combination as (1-product((1-prob\_i))),
where prob\_i is the probability of responding to the i-th cytokine
combination and the product is taken over all cytokine combinations
except the all-negative cytokine combination.
\end{Details}
%
\begin{Value}
A dataframe with columns sampleid and prob.
\end{Value}
%
\begin{Examples}
\begin{ExampleCode}
data("c_obj", package = "UtilsCompassSV")
response_prob(c_obj = c_obj)
response_prob(
  c_obj = c_obj,
  exc = c(
    "IFNg&IL2&TNF&!IL17&!IL6&!IL22",
    "IFNg&!IL2&TNF&!IL17&!IL6&!IL22"
  )
)
\end{ExampleCode}
\end{Examples}
\HeaderA{\Rpercent{}>\Rpercent{}}{Pipe operator}{.Rpcent.>.Rpcent.}
\keyword{internal}{\Rpercent{}>\Rpercent{}}
%
\begin{Description}
See \code{magrittr::\LinkA{\Rpercent{}>\Rpercent{}}{.Rpcent.>.Rpcent.}} for details.
\end{Description}
%
\begin{Usage}
\begin{verbatim}
lhs %>% rhs
\end{verbatim}
\end{Usage}
%
\begin{Arguments}
\begin{ldescription}
\item[\code{lhs}] A value or the magrittr placeholder.

\item[\code{rhs}] A function call using the magrittr semantics.
\end{ldescription}
\end{Arguments}
%
\begin{Value}
The result of calling `rhs(lhs)`.
\end{Value}
\printindex{}
\end{document}
